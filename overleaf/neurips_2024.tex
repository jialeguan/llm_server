\documentclass{article}

% if you need to pass options to natbib, use, e.g.:
%     \PassOptionsToPackage{numbers, compress}{natbib}
% before loading neurips_2024


% ready for submission
\usepackage{neurips_2024}

% to compile a preprint version, e.g., for submission to arXiv, add add the
% [preprint] option:
%     \usepackage[preprint]{neurips_2024}


% to compile a camera-ready version, add the [final] option, e.g.:
%     \usepackage[final]{neurips_2024}


% to avoid loading the natbib package, add option nonatbib:
%    \usepackage[nonatbib]{neurips_2024}


\usepackage[utf8]{inputenc} % allow utf-8 input
\usepackage[T1]{fontenc}    % use 8-bit T1 fonts
\usepackage{hyperref}       % hyperlinks
\usepackage{url}            % simple URL typesetting
\usepackage{booktabs}       % professional-quality tables
\usepackage{amsfonts}       % blackboard math symbols
\usepackage{nicefrac}       % compact symbols for 1/2, etc.
\usepackage{microtype}      % microtypography
\usepackage{xcolor}         % colors
\usepackage{amsmath}
\usepackage{tikz}
\usepackage{forest}

\title{An Empirical Study on Private Inference for\\
 Large Language Models}


% The \author macro works with any number of authors. There are two commands
% used to separate the names and addresses of multiple authors: \And and \AND.
%
% Using \And between authors leaves it to LaTeX to determine where to break the
% lines. Using \AND forces a line break at that point. So, if LaTeX puts 3 of 4
% authors names on the first line, and the last on the second line, try using
% \AND instead of \And before the third author name.


\author{%
  David S.~Hippocampus\thanks{Use footnote for providing further information
    about author (webpage, alternative address)---\emph{not} for acknowledging
    funding agencies.} \\
  Department of Computer Science\\
  Cranberry-Lemon University\\
  Pittsburgh, PA 15213 \\
  \texttt{hippo@cs.cranberry-lemon.edu} \\
  % examples of more authors
  % \And
  % Coauthor \\
  % Affiliation \\
  % Address \\
  % \texttt{email} \\
  % \AND
  % Coauthor \\
  % Affiliation \\
  % Address \\
  % \texttt{email} \\
  % \And
  % Coauthor \\
  % Affiliation \\
  % Address \\
  % \texttt{email} \\
  % \And
  % Coauthor \\
  % Affiliation \\
  % Address \\
  % \texttt{email} \\
}


\begin{document}

\maketitle

\begin{abstract}
    The abstract paragraph should be indented \nicefrac{1}{2}~inch (3~picas) on
    both the left- and right-hand margins. Use 10~point type, with a vertical
    spacing (leading) of 11~points.  The word \textbf{Abstract} must be centered,
    bold, and in point size 12. Two line spaces precede the abstract. The abstract
    must be limited to one paragraph.
  \end{abstract}

\section{Introduction}
\section{Background}

\subsection{LLM}

Language modeling, as the fundamental function of language models (LMs), involves modeling the likelihood of
the word sequence and predicting the distribution of subsequent words. Over recent years, researchers have discovered
that scaling up language models not only enhances their
language modeling ability but also engenders emergent
capabilities for tackling more intricate tasks beyond conven-
tional NLP tasks [25]. These scaled-up language models are
referred to as large language models (LLMs).

The mainstream LLMs are designed based on the Transformer architecture [26]. Specifically, a typical Transformer architecture is composed of several stacked Transformer
blocks. Typically, a Transformer block consists of a Multi-Head Self-Attention (MHSA) block, a Feed Forward Net-
work (FFN), and a LayerNorm (LN) operation. For each
block, it receives the output features of the previous one
as the input, and passes the features through each sub-
module to obtain the output. Specially, before the first block,
a tokenizer is used to convert the original input sentence
into a sequence of tokens, and a following embedding layer
serves to convert the tokens into the input features. Then,
the additional position embeddings are added into the input
features to encode the sequential order of each input token.

The core concept of the Transformer architecture is the
self-attention mechanism, which is adopted in the MHSA
block. Specifically, denoted the input features as $X=[x_1,x_2,\dots,x_n]$, the MHSA block applies linear projection to
them and obtains a set of queries Q, keys K and values V as

\begin{equation*}
    Q_i=XW^{Q_i}, K_i=XW^{K_i}, V_i=XW^{V_i},
\end{equation*}

where $W^{Q_i}$, $W^{K_i}$ and $W^{V_i}$ are the projection matrices
corresponding to the $i$-th attention head. Then the self-
attention operation is applied to each tuple of $(Q_i,K_i,V_i)$ and get the feature of the $i$-th attention head $Z_i$ as:

\begin{equation*}
    Z_i = \text{softmax}\frac{Q_iK_i^T}{\sqrt{d_k}}V_i,
\end{equation*}

where $d_k$ is the dimension of the key vectors. Note that
the self-attention operation contains the matrix multipli-
cation operation, its computation complexity is quadratic
in the input length. Finally, the MHSA block concatenates
the features of all the attention heads and applies a linear
projection to them to form its output $Z$ as:

\begin{equation*}
    Z = \text{Concat}(Z_1,Z_2,\dots,Z_h)W^O,
\end{equation*}

where $h$ is the number of attention heads and $W^O$ is the projection. As can be seen, the
self-attention mechanism allows the model to identify the
importance of different input parts regardless of the dis-
tance, and thus can capture the long-range dependencies
and complex relationships in the input sentence.

Another important module in the Transformer block is
the FFN. Typically, FFN is placed after the MHSA block
and consists of two linear transformation layers with a non-
linear activation function. It receives the output features X
from the MHSA block and processes them as

\begin{equation*}
    \text{FFN}(X) = \text{ReLU}(XW_1+b_1)W_2+b_2,
\end{equation*}

where $W_1$, $W_2$, $b_1$ and $b_2$ are the learnable parameters

\subsection{LLM Inference}
The most commonly used models for tasks like text generation are decoder-only Language Models (LLMs), which typically employ an auto-regressive mechanism to generate output sequences token by token. In an auto-regressive framework, each token is generated by considering the tokens that have already been generated, along with the input sequence. However, as the length of the sequence increases, the computational cost of generating each token grows rapidly, making the process time-consuming.

To mitigate this issue, a crucial optimization technique known as the Key-Value (KV) Cache has been introduced. KV caching works by storing and reusing previously computed key-value pairs from the attention mechanism within the Multi-Head Self-Attention (MHSA) block. This allows the model to avoid recalculating certain components of the attention mechanism, thereby reducing the latency of generation. LLM inference systems heavily rely on KV caching to improve efficiency, enabling faster token generation without sacrificing the quality of the output.

The LLM inference process can be divided into two main stages based on how the KV cache is utilized: the Prefill stage and the Decode stage.

In Prefill stage, the LLM prefills the cache with the key-value pairs of the input tokens. Specifically, the LLM first processes the input tokens through the embedding layer and the position embedding layer to obtain the input features. Then, the LLM passes the input features through the MHSA block to generate the key-value pairs. Finally, the LLM stores the key-value pairs in the cache.

In Decode stage, the LLM generates the output tokens based on the input tokens and the prefilled cache. Specifically, the LLM first processes the input tokens through the embedding layer and the position embedding layer to obtain the input features. Then, the LLM passes the input features through the MHSA block. During the self-attention operation, the LLM retrieves the key-value pairs from the cache and uses them to calculate the attention scores. Finally, the LLM generates the output tokens based on the attention scores.

\definecolor{ngreen}{HTML}{D5E8D4}
\definecolor{nblue}{HTML}{DAE8FC}
\definecolor{npurple}{HTML}{E1D5E7}

\begin{figure*}[h]
    \centering
    \tikzset{
        basic/.style  = {draw, text width=2cm, align=center, font=\sffamily, rectangle},
        root/.style   = {basic, rounded corners=2pt, thin, align=center, fill=white,text width=8cm, rotate=90, font=\footnotesize},
        dnode/.style = {basic, thin, rounded corners=2pt, align=center, fill=ngreen,text width=3.5cm, font=\footnotesize},
        dnode_1/.style = {basic, thin, rounded corners=2pt, align=center, fill=ngreen,text width=2.5cm, font=\footnotesize},
        mnode/.style = {basic, thin, rounded corners=2pt, align=center, fill=nblue, text width=3.5cm, font=\footnotesize},
        mnode_1/.style = {basic, thin, rounded corners=2pt, align=center, fill=nblue, text width=2.5cm, font=\footnotesize},
        snode/.style = {basic, thin, rounded corners=2pt, align=center, fill=npurple,text width=3.5cm, font=\footnotesize},
        snode_1/.style = {basic, thin, rounded corners=2pt, align=center, fill=npurple,text width=2.5cm, font=\footnotesize},
        tnode/.style = {basic, thin, align=left, fill=pink!60, text width=15em, align=center},
        xnode/.style = {basic, thin, rounded corners=2pt, align=center, fill=blue!20,text width=5cm,},
        wnode/.style = {basic, thin, align=left, fill=pink!10!blue!80!red!10, text width=6.5em},
        %edge from parent/.style = {draw=black, edge from parent fork right}
        %edge from parent/.style = {draw=black, edge from parent fork down}
    }
    %
    \begin{forest}
        for tree={
        if level=0{
        grow=east,
        growth parent anchor=east,
        parent anchor=south,
        child anchor=west,
        edge path={\noexpand\path[\forestoption{edge},->, >={latex}]
        (!u.parent anchor) -- +(5pt,0pt) |- (.child anchor)
        \forestoption{edge label};},
        }
        {
        grow=east,
        growth parent anchor=east,
        parent anchor=east,
        child anchor=west,
        edge path={\noexpand\path[\forestoption{edge},->, >={latex}]
        (!u.parent anchor) -- +(5pt,0pt) |- (.child anchor)
        \forestoption{edge label};},
        }
        }
        % l sep is used for arrow distance
        [Efficient Inference for Large Language Models, root
        [System-level Optimization (Sec.~\ref{sec:system-level-opt}), snode_1
        [Serving System \\ (Sec.~\ref{sec:serving_system}), snode
            [Distributed Systems, snode]
            [Scheduling, snode]
            [Batching, snode]
            [Memory Management, snode]
        ]
        [Inference Engine \\ (Sec.~\ref{sec:inference_engine}), snode
            %[Model Parallelism, snode]
            [Speculative Decoding, snode]
            [Offloading, snode]
            [Graph and Operator Optimization, snode]
        ]
        ]
        [Model-level Optimization (Sec.~\ref{sec:model-level-opt}), mnode_1
        [Model Compression \\ (Sec.~\ref{sec:model_compression}), mnode
            [Dynamic Inference, mnode]
            [Knowledge Distillation, mnode
                    [Black-box KD, mnode]
                    [White-box KD, mnode]
            ]
            [Structure Optimization, mnode
                    [Neural Architecture Search, mnode]
                    [Structure Factorization, mnode]
            ]
            [Sparsification, mnode
                    [Sparse Attention, mnode]
                    [Weight Pruning, mnode]
            ]
            [Quantization, mnode
                    [Quantization-aware Training, mnode]
                    [Post-Training Quantization, mnode]
            ]
        ]
        [Efficient Structure Design \\ (Sec.~\ref{sec:efficient_structure}), mnode
            [Transformer Alternate, mnode]
            [Efficient Attention Design, mnode
                    [Multi/Group-Query Attention, mnode]
                    [Low-Complexity Attention, mnode]
            ]
            [Efficient FFN Design, mnode]
        ]
        ]
        [Data-level Optimization (Sec.~\ref{sec:data-level-opt}), dnode_1
        [Output Organization \\ (Sec.~\ref{sec:output_compress}), dnode]
        [Input Compression \\ (Sec.~\ref{sec:input_compress}), dnode
            [Retrieval-Augmented Generation, dnode]
            [Soft Prompt-based Compression, dnode]
            [Prompt Summary, dnode]
            [Prompt Pruning, dnode]
        ]
        ]
        ]
    \end{forest}

    \caption{Taxonomy of efficient inference methods for Large Language Models.}
    \label{fig:framework}
\end{figure*}

\subsection{Private Cloud Compute}

The emergence of LLMs has introduced novel challenges in cloud computing, particularly in the realm of data privacy and security. Unlike traditional server-client architectures, which often maintain persistent user data and rely on policy-based privacy protections, LLM inference necessitates real-time access to unencrypted user requests and personal data, coupled with significantly higher computational demands. This paradigm shift exposes potential vulnerabilities in data handling and retention, as well as challenges in providing verifiable privacy guarantees and runtime transparency. To address these concerns, Apple has proposed Private Cloud Compute (PCC), a pioneering framework designed to extend device-level security into the cloud environment for AI processing. PCC establishes a set of stringent requirements that fundamentally reimagine cloud AI security: it mandates stateless computation on personal user data, ensuring that data is used solely for request fulfillment and is not retained post-processing; it demands technically enforceable guarantees that can be analyzed and constrained across all critical components; it eliminates privileged runtime access that could bypass privacy safeguards; it implements non-targetability to prevent attacks aimed at specific users; and it provides verifiable transparency, allowing security researchers to inspect and validate the system's integrity. These requirements collectively represent a significant departure from conventional cloud service security models, aiming to establish a new standard for secure and private AI processing in cloud environments.

\subsubsection{Taxonomy}

Disclosed privacy threats

\section{Optimization}
\label{sec:inference_engine}
\label{sec:model-level-opt}
\label{sec:system-level-opt}
Academic and industrial systems

\subsection{Memory Management}

\subsubsection{PagedAttention}

\subsubsection{Prefix Caching}

\subsection{Transmission}

\subsubsection{Data Duplication}

\subsubsection{Data Pulling}

\subsubsection{Request Migration}

\subsection{Scheduling}

\subsubsection{Priority-based Scheduling}

\subsubsection{Instance Flip}

\subsubsection{Global Profiling}

\subsection{Batch Processing}

\subsubsection{Iteration-level Batch}

\subsubsection{Chunked Prefill}

\subsubsection{Prepack Prefill}

\subsection{Parallel Processing}

\subsubsection{Speculative Inference}





\section{Threats}

% 从系统领域引入的威胁
\subsection{Threats from the System Domain}

% LLM自身的威胁
\subsection{Threats from the LLM System}
\section{Discussion}

Mitigation

Design Principle


\begin{ack}
Use unnumbered first level headings for the acknowledgments. All acknowledgments
go at the end of the paper before the list of references. Moreover, you are required to declare
funding (financial activities supporting the submitted work) and competing interests (related financial activities outside the submitted work).
More information about this disclosure can be found at: \url{https://neurips.cc/Conferences/2024/PaperInformation/FundingDisclosure}.


Do {\bf not} include this section in the anonymized submission, only in the final paper. You can use the \texttt{ack} environment provided in the style file to automatically hide this section in the anonymized submission.
\end{ack}

\section*{References}


References follow the acknowledgments in the camera-ready paper. Use unnumbered first-level heading for
the references. Any choice of citation style is acceptable as long as you are
consistent. It is permissible to reduce the font size to \verb+small+ (9 point)
when listing the references.
Note that the Reference section does not count towards the page limit.
\medskip


{
\small


[1] Alexander, J.A.\ \& Mozer, M.C.\ (1995) Template-based algorithms for
connectionist rule extraction. In G.\ Tesauro, D.S.\ Touretzky and T.K.\ Leen
(eds.), {\it Advances in Neural Information Processing Systems 7},
pp.\ 609--616. Cambridge, MA: MIT Press.


[2] Bower, J.M.\ \& Beeman, D.\ (1995) {\it The Book of GENESIS: Exploring
  Realistic Neural Models with the GEneral NEural SImulation System.}  New York:
TELOS/Springer--Verlag.


[3] Hasselmo, M.E., Schnell, E.\ \& Barkai, E.\ (1995) Dynamics of learning and
recall at excitatory recurrent synapses and cholinergic modulation in rat
hippocampal region CA3. {\it Journal of Neuroscience} {\bf 15}(7):5249-5262.
}


%%%%%%%%%%%%%%%%%%%%%%%%%%%%%%%%%%%%%%%%%%%%%%%%%%%%%%%%%%%%

\appendix

\section{Appendix / supplemental material}


Optionally include supplemental material (complete proofs, additional experiments and plots) in appendix.
All such materials \textbf{SHOULD be included in the main submission.}

%%%%%%%%%%%%%%%%%%%%%%%%%%%%%%%%%%%%%%%%%%%%%%%%%%%%%%%%%%%%

\newpage
\section*{NeurIPS Paper Checklist}

%%% BEGIN INSTRUCTIONS %%%
The checklist is designed to encourage best practices for responsible machine learning research, addressing issues of reproducibility, transparency, research ethics, and societal impact. Do not remove the checklist: {\bf The papers not including the checklist will be desk rejected.} The checklist should follow the references and follow the (optional) supplemental material.  The checklist does NOT count towards the page
limit. 

Please read the checklist guidelines carefully for information on how to answer these questions. For each question in the checklist:
\begin{itemize}
    \item You should answer \answerYes{}, \answerNo{}, or \answerNA{}.
    \item \answerNA{} means either that the question is Not Applicable for that particular paper or the relevant information is Not Available.
    \item Please provide a short (1–2 sentence) justification right after your answer (even for NA). 
   % \item {\bf The papers not including the checklist will be desk rejected.}
\end{itemize}

{\bf The checklist answers are an integral part of your paper submission.} They are visible to the reviewers, area chairs, senior area chairs, and ethics reviewers. You will be asked to also include it (after eventual revisions) with the final version of your paper, and its final version will be published with the paper.

The reviewers of your paper will be asked to use the checklist as one of the factors in their evaluation. While "\answerYes{}" is generally preferable to "\answerNo{}", it is perfectly acceptable to answer "\answerNo{}" provided a proper justification is given (e.g., "error bars are not reported because it would be too computationally expensive" or "we were unable to find the license for the dataset we used"). In general, answering "\answerNo{}" or "\answerNA{}" is not grounds for rejection. While the questions are phrased in a binary way, we acknowledge that the true answer is often more nuanced, so please just use your best judgment and write a justification to elaborate. All supporting evidence can appear either in the main paper or the supplemental material, provided in appendix. If you answer \answerYes{} to a question, in the justification please point to the section(s) where related material for the question can be found.

IMPORTANT, please:
\begin{itemize}
    \item {\bf Delete this instruction block, but keep the section heading ``NeurIPS paper checklist"},
    \item  {\bf Keep the checklist subsection headings, questions/answers and guidelines below.}
    \item {\bf Do not modify the questions and only use the provided macros for your answers}.
\end{itemize} 
 

%%% END INSTRUCTIONS %%%


\begin{enumerate}

\item {\bf Claims}
    \item[] Question: Do the main claims made in the abstract and introduction accurately reflect the paper's contributions and scope?
    \item[] Answer: \answerTODO{} % Replace by \answerYes{}, \answerNo{}, or \answerNA{}.
    \item[] Justification: \justificationTODO{}
    \item[] Guidelines:
    \begin{itemize}
        \item The answer NA means that the abstract and introduction do not include the claims made in the paper.
        \item The abstract and/or introduction should clearly state the claims made, including the contributions made in the paper and important assumptions and limitations. A No or NA answer to this question will not be perceived well by the reviewers. 
        \item The claims made should match theoretical and experimental results, and reflect how much the results can be expected to generalize to other settings. 
        \item It is fine to include aspirational goals as motivation as long as it is clear that these goals are not attained by the paper. 
    \end{itemize}

\item {\bf Limitations}
    \item[] Question: Does the paper discuss the limitations of the work performed by the authors?
    \item[] Answer: \answerTODO{} % Replace by \answerYes{}, \answerNo{}, or \answerNA{}.
    \item[] Justification: \justificationTODO{}
    \item[] Guidelines:
    \begin{itemize}
        \item The answer NA means that the paper has no limitation while the answer No means that the paper has limitations, but those are not discussed in the paper. 
        \item The authors are encouraged to create a separate "Limitations" section in their paper.
        \item The paper should point out any strong assumptions and how robust the results are to violations of these assumptions (e.g., independence assumptions, noiseless settings, model well-specification, asymptotic approximations only holding locally). The authors should reflect on how these assumptions might be violated in practice and what the implications would be.
        \item The authors should reflect on the scope of the claims made, e.g., if the approach was only tested on a few datasets or with a few runs. In general, empirical results often depend on implicit assumptions, which should be articulated.
        \item The authors should reflect on the factors that influence the performance of the approach. For example, a facial recognition algorithm may perform poorly when image resolution is low or images are taken in low lighting. Or a speech-to-text system might not be used reliably to provide closed captions for online lectures because it fails to handle technical jargon.
        \item The authors should discuss the computational efficiency of the proposed algorithms and how they scale with dataset size.
        \item If applicable, the authors should discuss possible limitations of their approach to address problems of privacy and fairness.
        \item While the authors might fear that complete honesty about limitations might be used by reviewers as grounds for rejection, a worse outcome might be that reviewers discover limitations that aren't acknowledged in the paper. The authors should use their best judgment and recognize that individual actions in favor of transparency play an important role in developing norms that preserve the integrity of the community. Reviewers will be specifically instructed to not penalize honesty concerning limitations.
    \end{itemize}

\item {\bf Theory Assumptions and Proofs}
    \item[] Question: For each theoretical result, does the paper provide the full set of assumptions and a complete (and correct) proof?
    \item[] Answer: \answerTODO{} % Replace by \answerYes{}, \answerNo{}, or \answerNA{}.
    \item[] Justification: \justificationTODO{}
    \item[] Guidelines:
    \begin{itemize}
        \item The answer NA means that the paper does not include theoretical results. 
        \item All the theorems, formulas, and proofs in the paper should be numbered and cross-referenced.
        \item All assumptions should be clearly stated or referenced in the statement of any theorems.
        \item The proofs can either appear in the main paper or the supplemental material, but if they appear in the supplemental material, the authors are encouraged to provide a short proof sketch to provide intuition. 
        \item Inversely, any informal proof provided in the core of the paper should be complemented by formal proofs provided in appendix or supplemental material.
        \item Theorems and Lemmas that the proof relies upon should be properly referenced. 
    \end{itemize}

    \item {\bf Experimental Result Reproducibility}
    \item[] Question: Does the paper fully disclose all the information needed to reproduce the main experimental results of the paper to the extent that it affects the main claims and/or conclusions of the paper (regardless of whether the code and data are provided or not)?
    \item[] Answer: \answerTODO{} % Replace by \answerYes{}, \answerNo{}, or \answerNA{}.
    \item[] Justification: \justificationTODO{}
    \item[] Guidelines:
    \begin{itemize}
        \item The answer NA means that the paper does not include experiments.
        \item If the paper includes experiments, a No answer to this question will not be perceived well by the reviewers: Making the paper reproducible is important, regardless of whether the code and data are provided or not.
        \item If the contribution is a dataset and/or model, the authors should describe the steps taken to make their results reproducible or verifiable. 
        \item Depending on the contribution, reproducibility can be accomplished in various ways. For example, if the contribution is a novel architecture, describing the architecture fully might suffice, or if the contribution is a specific model and empirical evaluation, it may be necessary to either make it possible for others to replicate the model with the same dataset, or provide access to the model. In general. releasing code and data is often one good way to accomplish this, but reproducibility can also be provided via detailed instructions for how to replicate the results, access to a hosted model (e.g., in the case of a large language model), releasing of a model checkpoint, or other means that are appropriate to the research performed.
        \item While NeurIPS does not require releasing code, the conference does require all submissions to provide some reasonable avenue for reproducibility, which may depend on the nature of the contribution. For example
        \begin{enumerate}
            \item If the contribution is primarily a new algorithm, the paper should make it clear how to reproduce that algorithm.
            \item If the contribution is primarily a new model architecture, the paper should describe the architecture clearly and fully.
            \item If the contribution is a new model (e.g., a large language model), then there should either be a way to access this model for reproducing the results or a way to reproduce the model (e.g., with an open-source dataset or instructions for how to construct the dataset).
            \item We recognize that reproducibility may be tricky in some cases, in which case authors are welcome to describe the particular way they provide for reproducibility. In the case of closed-source models, it may be that access to the model is limited in some way (e.g., to registered users), but it should be possible for other researchers to have some path to reproducing or verifying the results.
        \end{enumerate}
    \end{itemize}


\item {\bf Open access to data and code}
    \item[] Question: Does the paper provide open access to the data and code, with sufficient instructions to faithfully reproduce the main experimental results, as described in supplemental material?
    \item[] Answer: \answerTODO{} % Replace by \answerYes{}, \answerNo{}, or \answerNA{}.
    \item[] Justification: \justificationTODO{}
    \item[] Guidelines:
    \begin{itemize}
        \item The answer NA means that paper does not include experiments requiring code.
        \item Please see the NeurIPS code and data submission guidelines (\url{https://nips.cc/public/guides/CodeSubmissionPolicy}) for more details.
        \item While we encourage the release of code and data, we understand that this might not be possible, so “No” is an acceptable answer. Papers cannot be rejected simply for not including code, unless this is central to the contribution (e.g., for a new open-source benchmark).
        \item The instructions should contain the exact command and environment needed to run to reproduce the results. See the NeurIPS code and data submission guidelines (\url{https://nips.cc/public/guides/CodeSubmissionPolicy}) for more details.
        \item The authors should provide instructions on data access and preparation, including how to access the raw data, preprocessed data, intermediate data, and generated data, etc.
        \item The authors should provide scripts to reproduce all experimental results for the new proposed method and baselines. If only a subset of experiments are reproducible, they should state which ones are omitted from the script and why.
        \item At submission time, to preserve anonymity, the authors should release anonymized versions (if applicable).
        \item Providing as much information as possible in supplemental material (appended to the paper) is recommended, but including URLs to data and code is permitted.
    \end{itemize}


\item {\bf Experimental Setting/Details}
    \item[] Question: Does the paper specify all the training and test details (e.g., data splits, hyperparameters, how they were chosen, type of optimizer, etc.) necessary to understand the results?
    \item[] Answer: \answerTODO{} % Replace by \answerYes{}, \answerNo{}, or \answerNA{}.
    \item[] Justification: \justificationTODO{}
    \item[] Guidelines:
    \begin{itemize}
        \item The answer NA means that the paper does not include experiments.
        \item The experimental setting should be presented in the core of the paper to a level of detail that is necessary to appreciate the results and make sense of them.
        \item The full details can be provided either with the code, in appendix, or as supplemental material.
    \end{itemize}

\item {\bf Experiment Statistical Significance}
    \item[] Question: Does the paper report error bars suitably and correctly defined or other appropriate information about the statistical significance of the experiments?
    \item[] Answer: \answerTODO{} % Replace by \answerYes{}, \answerNo{}, or \answerNA{}.
    \item[] Justification: \justificationTODO{}
    \item[] Guidelines:
    \begin{itemize}
        \item The answer NA means that the paper does not include experiments.
        \item The authors should answer "Yes" if the results are accompanied by error bars, confidence intervals, or statistical significance tests, at least for the experiments that support the main claims of the paper.
        \item The factors of variability that the error bars are capturing should be clearly stated (for example, train/test split, initialization, random drawing of some parameter, or overall run with given experimental conditions).
        \item The method for calculating the error bars should be explained (closed form formula, call to a library function, bootstrap, etc.)
        \item The assumptions made should be given (e.g., Normally distributed errors).
        \item It should be clear whether the error bar is the standard deviation or the standard error of the mean.
        \item It is OK to report 1-sigma error bars, but one should state it. The authors should preferably report a 2-sigma error bar than state that they have a 96\% CI, if the hypothesis of Normality of errors is not verified.
        \item For asymmetric distributions, the authors should be careful not to show in tables or figures symmetric error bars that would yield results that are out of range (e.g. negative error rates).
        \item If error bars are reported in tables or plots, The authors should explain in the text how they were calculated and reference the corresponding figures or tables in the text.
    \end{itemize}

\item {\bf Experiments Compute Resources}
    \item[] Question: For each experiment, does the paper provide sufficient information on the computer resources (type of compute workers, memory, time of execution) needed to reproduce the experiments?
    \item[] Answer: \answerTODO{} % Replace by \answerYes{}, \answerNo{}, or \answerNA{}.
    \item[] Justification: \justificationTODO{}
    \item[] Guidelines:
    \begin{itemize}
        \item The answer NA means that the paper does not include experiments.
        \item The paper should indicate the type of compute workers CPU or GPU, internal cluster, or cloud provider, including relevant memory and storage.
        \item The paper should provide the amount of compute required for each of the individual experimental runs as well as estimate the total compute. 
        \item The paper should disclose whether the full research project required more compute than the experiments reported in the paper (e.g., preliminary or failed experiments that didn't make it into the paper). 
    \end{itemize}
    
\item {\bf Code Of Ethics}
    \item[] Question: Does the research conducted in the paper conform, in every respect, with the NeurIPS Code of Ethics \url{https://neurips.cc/public/EthicsGuidelines}?
    \item[] Answer: \answerTODO{} % Replace by \answerYes{}, \answerNo{}, or \answerNA{}.
    \item[] Justification: \justificationTODO{}
    \item[] Guidelines:
    \begin{itemize}
        \item The answer NA means that the authors have not reviewed the NeurIPS Code of Ethics.
        \item If the authors answer No, they should explain the special circumstances that require a deviation from the Code of Ethics.
        \item The authors should make sure to preserve anonymity (e.g., if there is a special consideration due to laws or regulations in their jurisdiction).
    \end{itemize}


\item {\bf Broader Impacts}
    \item[] Question: Does the paper discuss both potential positive societal impacts and negative societal impacts of the work performed?
    \item[] Answer: \answerTODO{} % Replace by \answerYes{}, \answerNo{}, or \answerNA{}.
    \item[] Justification: \justificationTODO{}
    \item[] Guidelines:
    \begin{itemize}
        \item The answer NA means that there is no societal impact of the work performed.
        \item If the authors answer NA or No, they should explain why their work has no societal impact or why the paper does not address societal impact.
        \item Examples of negative societal impacts include potential malicious or unintended uses (e.g., disinformation, generating fake profiles, surveillance), fairness considerations (e.g., deployment of technologies that could make decisions that unfairly impact specific groups), privacy considerations, and security considerations.
        \item The conference expects that many papers will be foundational research and not tied to particular applications, let alone deployments. However, if there is a direct path to any negative applications, the authors should point it out. For example, it is legitimate to point out that an improvement in the quality of generative models could be used to generate deepfakes for disinformation. On the other hand, it is not needed to point out that a generic algorithm for optimizing neural networks could enable people to train models that generate Deepfakes faster.
        \item The authors should consider possible harms that could arise when the technology is being used as intended and functioning correctly, harms that could arise when the technology is being used as intended but gives incorrect results, and harms following from (intentional or unintentional) misuse of the technology.
        \item If there are negative societal impacts, the authors could also discuss possible mitigation strategies (e.g., gated release of models, providing defenses in addition to attacks, mechanisms for monitoring misuse, mechanisms to monitor how a system learns from feedback over time, improving the efficiency and accessibility of ML).
    \end{itemize}
    
\item {\bf Safeguards}
    \item[] Question: Does the paper describe safeguards that have been put in place for responsible release of data or models that have a high risk for misuse (e.g., pretrained language models, image generators, or scraped datasets)?
    \item[] Answer: \answerTODO{} % Replace by \answerYes{}, \answerNo{}, or \answerNA{}.
    \item[] Justification: \justificationTODO{}
    \item[] Guidelines:
    \begin{itemize}
        \item The answer NA means that the paper poses no such risks.
        \item Released models that have a high risk for misuse or dual-use should be released with necessary safeguards to allow for controlled use of the model, for example by requiring that users adhere to usage guidelines or restrictions to access the model or implementing safety filters. 
        \item Datasets that have been scraped from the Internet could pose safety risks. The authors should describe how they avoided releasing unsafe images.
        \item We recognize that providing effective safeguards is challenging, and many papers do not require this, but we encourage authors to take this into account and make a best faith effort.
    \end{itemize}

\item {\bf Licenses for existing assets}
    \item[] Question: Are the creators or original owners of assets (e.g., code, data, models), used in the paper, properly credited and are the license and terms of use explicitly mentioned and properly respected?
    \item[] Answer: \answerTODO{} % Replace by \answerYes{}, \answerNo{}, or \answerNA{}.
    \item[] Justification: \justificationTODO{}
    \item[] Guidelines:
    \begin{itemize}
        \item The answer NA means that the paper does not use existing assets.
        \item The authors should cite the original paper that produced the code package or dataset.
        \item The authors should state which version of the asset is used and, if possible, include a URL.
        \item The name of the license (e.g., CC-BY 4.0) should be included for each asset.
        \item For scraped data from a particular source (e.g., website), the copyright and terms of service of that source should be provided.
        \item If assets are released, the license, copyright information, and terms of use in the package should be provided. For popular datasets, \url{paperswithcode.com/datasets} has curated licenses for some datasets. Their licensing guide can help determine the license of a dataset.
        \item For existing datasets that are re-packaged, both the original license and the license of the derived asset (if it has changed) should be provided.
        \item If this information is not available online, the authors are encouraged to reach out to the asset's creators.
    \end{itemize}

\item {\bf New Assets}
    \item[] Question: Are new assets introduced in the paper well documented and is the documentation provided alongside the assets?
    \item[] Answer: \answerTODO{} % Replace by \answerYes{}, \answerNo{}, or \answerNA{}.
    \item[] Justification: \justificationTODO{}
    \item[] Guidelines:
    \begin{itemize}
        \item The answer NA means that the paper does not release new assets.
        \item Researchers should communicate the details of the dataset/code/model as part of their submissions via structured templates. This includes details about training, license, limitations, etc. 
        \item The paper should discuss whether and how consent was obtained from people whose asset is used.
        \item At submission time, remember to anonymize your assets (if applicable). You can either create an anonymized URL or include an anonymized zip file.
    \end{itemize}

\item {\bf Crowdsourcing and Research with Human Subjects}
    \item[] Question: For crowdsourcing experiments and research with human subjects, does the paper include the full text of instructions given to participants and screenshots, if applicable, as well as details about compensation (if any)? 
    \item[] Answer: \answerTODO{} % Replace by \answerYes{}, \answerNo{}, or \answerNA{}.
    \item[] Justification: \justificationTODO{}
    \item[] Guidelines:
    \begin{itemize}
        \item The answer NA means that the paper does not involve crowdsourcing nor research with human subjects.
        \item Including this information in the supplemental material is fine, but if the main contribution of the paper involves human subjects, then as much detail as possible should be included in the main paper. 
        \item According to the NeurIPS Code of Ethics, workers involved in data collection, curation, or other labor should be paid at least the minimum wage in the country of the data collector. 
    \end{itemize}

\item {\bf Institutional Review Board (IRB) Approvals or Equivalent for Research with Human Subjects}
    \item[] Question: Does the paper describe potential risks incurred by study participants, whether such risks were disclosed to the subjects, and whether Institutional Review Board (IRB) approvals (or an equivalent approval/review based on the requirements of your country or institution) were obtained?
    \item[] Answer: \answerTODO{} % Replace by \answerYes{}, \answerNo{}, or \answerNA{}.
    \item[] Justification: \justificationTODO{}
    \item[] Guidelines:
    \begin{itemize}
        \item The answer NA means that the paper does not involve crowdsourcing nor research with human subjects.
        \item Depending on the country in which research is conducted, IRB approval (or equivalent) may be required for any human subjects research. If you obtained IRB approval, you should clearly state this in the paper. 
        \item We recognize that the procedures for this may vary significantly between institutions and locations, and we expect authors to adhere to the NeurIPS Code of Ethics and the guidelines for their institution. 
        \item For initial submissions, do not include any information that would break anonymity (if applicable), such as the institution conducting the review.
    \end{itemize}

\end{enumerate}


\end{document}